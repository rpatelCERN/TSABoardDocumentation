\documentclass[12pt,a4paper]{article}
\usepackage[utf8]{inputenc}
\usepackage{amsmath}
\usepackage{amsfonts}
\usepackage{amssymb}
\author{TSA Board Collaboration}
\title{Track Standalone Algorithms}
\begin{document}
\section{Track Standalone Algorithms}

Standalone track algorithms represent an important part of the Phase-II trigger menu, since they provide the ability to robustly use an independent sub-detector. This section focuses on standalone algorithms with an input collection of L1 tracks as described in Section~\ref{}. The tracker is the most granular layer in the CMS L1 decision, so is key for delivering pile-up resilient measurements at L1 which include the primary vertex z-position, track-based jets, and track based missing energy. The harsh pile-up conditions of the Phase-2 LHC make it paramount to seperate L1tracks originating from the primary collision from spurious track hits caused by the large occupancy in the tracking volume. Track purity requirements become a key component to achieve manageable data-recording rates while maintaining low trigger thresholds. Furthermore, the tracks provide the primary vertex z-position, which greatly will mitigate the pileup effects from 200 other proton interactions in the global trigger which combines sub-detector information.

This section focuses on the core trigger alogrithms running on the Track-Standalone-Algorithm (TSA) FPGA boards. This board consolidates tracks from the full detector range, and provides a primary vertex, a collection of track-based jets, and track-based missing energy with minimal latency. Figure~\ref{} shows a block diagram of the TSA board. The board consists of a track interface section to read in all tracks and parse the track word from different links. The tracks are then sent to the core algorithms on the board, which are described in this section. The output L1 objects are then received downstream for the L1 decision. Section~\ref{sec:FastHisto} will describe the firmware implementation of fast vertexing algorithm originally described in~\cite{TP} along with the resource utilization on the TSA board and the latency. Section~\ref{sec:TrkJets} describes a track clustering algorithm, the FPGA resource usage, and the performance of track-based jet triggers. Section~\ref{sec:TkMET} describes the track-based missing energy algorithm which takes the primary z-vertex as input and computes the missing energy transverse to the beamline from L1 tracks within a window of the primary vertex. Section~\ref{sec:DispTracks} shows the performance of including extending tracking in the jet algorithms, which allows jets to be tagged as displaced for dedicated BSM long-lived particle triggers.

\subsection{Vertexing Firmware }


\subsection{Jet Algorithm with Tracks}
\input{TrackMET}
\subsection{Displaced Jets}
\label{sec:DispTracks}


\end{document}