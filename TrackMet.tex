\subsection{Track-based Missing Energy}
\label{sec:TkMET}

The vector sum of the transverse momenta (MET) of all particles produced in the primary collision is a key input for BSM triggers at Level 1. For the track-based algorithm described in this section, one of the main challenges is excluding tracks from bad combinations, ``fake tracks," that give high transverse momentum. Although these tracks are rare in pileup after requiring a tight window around the primary vertex z position, events containing these tracks would dominate an L1 MET trigger. The algorithm makes use of track purity requirements in addition to the $\Delta z\left(PV_{z}, trk_{z}\right)$ requirement to reduce the number of events with poorly measured momentum balance.

<<<<<<< HEAD
The track purity selection is based mainly on the confines of the detector, the track reconstruction algoirthm, and  the available track fit quality parameters. The track $p_{T}$ and $\eta$ requirements shown in Table~\ref{tab:trkpurity} are based on the minimal $p_{T}$ of tracks that can be reconstructed reliably and the tracking acceptance. The minimal number of track layers is also based on the minimal requirement for the track fit. There is a further requirement that 4 of the track layers must also have stubs to remove any 4-stub tracks that were created with only hits in 3 layers. Requirements on the track fit quality measured as $\chi^{2}_{n.dof}$ is set to a minimum of 50 to reduce tracks that are poorly reconstructed. To further reject high momentum tracks from poorly reconstructed track hit combinations, a new variable, bend consistency, $\chi^{2}_{bend}$ is measured. This measure of the bend of the track is uncorrelated to the $\chi^{2}_{n.dof}$ from the track fit. $\chi^{2}_{bend}$ is calculated based on the horizontal distance between the two consecutive track hits in a track module, the stubs. The bend consistency quantifys how compatible the bend of the track hits measured in the detector is compatible with the reconstructed track $p_{T}$. A bad combination of track hits tend to have a large value of bend consistency compared to well-reconstructed tracks. The cut on $\chi^{2}_{bend}$ is optimized to lower the fraction of poorly reconstructed tracks that are included in the MET caculation as well as maintaining a reasonable L1 data-rate for a track-based MET trigger. [Here you would reference the fake-rate and the L1 Rate Reduction plot]
\begin{table}[h]
\begin{tabular}{|c|c|}
Track Variable & Cut\\
$N_{stubs}$ per track layer&$\geq 4$ \\
$p_{T}$ & 2~GeV \\
$\chi^{2}_{n.dof}$ & 50 \\
$\chi^{2}_{bend}$  & 1.75 \\

\end{tabular}
\caption{ L1 track purity requirements for tracks input to both the MET computation and the L1 track-based jet clustering. Cuts are optimized to preserve a sharp-on turn on in track-based MET, $H_{T}$ and $H^{miss}_{T}$ rejecting the bulk of "fake" tracks from bad combination of track hits.}
\label{tab:trkpurity}
\end{table}

The L1 vertex described in the TSA firmware algorithm in Section~\ref{sec:FastHisto} is input to the L1 track-based MET algorithm. A window of $\Delta z\left(PV_{z}, trk_{z}\right)$ is required for pseudo-rapidity regions within the tracking acceptance. Table~\ref{tab:zwindows} shows the range of values for the allowed range of track-z around around the measured primary vertex z. Central tracks allow for a tight window of 0.4cm while more forward tracks need a wider z-window ( as large as 2.2cm) to include tracks within $3\sigma$ of the gaussian track z-resolution in the forward regions.
\begin{table}[h]
\begin{tabular}{|c|c|}
$\eta$ range & $\Delta z\left(PV_{z}, trk_{z}\right)$\\
\end{tabular}

\caption{Window around the primary vertex optimized for each pseudorapidy range. }
\label{tab:zwindows}
\end{table}
=======
The track purity selection (Table \ref{tab:tkMETcuts}) is based mainly on the confines of the detector, the track reconstruction algorithm, and the available track fit quality parameters. The track $p_{T}$ and $\eta$ requirements shown in Table ~\ref{tab:tkMETcuts} are based on the minimal $p_{T}$ of tracks that can be reconstructed reliably and the tracking acceptance. The minimal number of track layers is also based on the minimal requirement for the track fit. There is a further requirement that four of the track layers must also have stubs to remove any 4-stub tracks that were created with hits in only three layers. A requirement on the track fit quality measured as $\chi^{2}_{n.dof}$ is set to a minimum of 50 to reduce poorly reconstructed tracks. To further reject high momentum fake tracks, a new variable, bend chi-squared, $\chi^{2}_{bend}$, is measured. This measure of the bend of the track is uncorrelated to the $\chi^{2}_{n.dof}$ from the track fit, as the track fit only relies on one hit per module, and the $\chi^{2}_{bend}$ is calculated from both hits. $\chi^{2}_{bend}$ is calculated based on the horizontal distance between the two consecutive track hits in a track module, the stubs. The bend chi-squared quantifies how compatible the bend of the track hits measured in the detector is with the reconstructed track $p_{T}$. A bad combination of track hits tends to have a large value of bend chi-squared compared to well-reconstructed tracks. The cut on $\chi^{2}_{bend}$ is optimized to lower the fraction of poorly reconstructed tracks that are included in the MET calculation, as well as maintain a reasonable L1 data-rate for a track-based MET trigger. [Here you would reference the fake-rate and the L1 Rate Reduction plot]

The L1 vertex described in the TSA firmware algorithm in Section~\ref{sec:FastHisto} is an input for the L1 track-based MET algorithm. A window of $\Delta z\left(PV_{z}, trk_{z}\right)$ is required for pseudo-rapidity regions within the tracking acceptance. Table~\ref{tab:tkMETDeltaz} shows the range of values for the allowed range of track-z around the measured primary vertex z. Central tracks allow for a tight window of 0.4 cm, while more forward tracks need a wider z-window (as large as 2.2 cm) to include tracks within $3\sigma$ of the Gaussian track z-resolution.

Including these criteria for the optimized track MET measurement reduces the threshold at a data-taking rate of 35 kHz from 235 GeV to 60 GeV. This threshold aligns closely to the simulated true track MET performance. The threshold of 60 GeV gives 95$\%$ efficiency for the trigger turn-on at 300 GeV in generator-level MET.


\begin{table}
\centering
\label{tab:tkMETcuts}
\caption{List of cuts optimized for track MET.}
\begin{tabular}{ |c|c| }
\hline
p$_\textrm{T}$  & $>$ 2 GeV \\
p$_\textrm{T}$ & saturate at 200 GeV \\
$|\eta|$  & $<$ 2.4 \\
Num. stubs  & $\geq$ 4 \\
$|\textrm{z}_{0}|$  & $<$ 15 cm \\
$|\Delta \textrm{z}|$  & by track $\eta$ \\
$\chi^{2}/\textrm{dof}$ & $<$ 50 \\
$\chi^{2}_{\textrm{bend}}$ & $<$ 1.75 \\
\hline
\end{tabular}
\end{table}

\begin{table}
\centering
\label{tab:tkMETDeltaz}
\caption{$\Delta$z cut based on track $\eta$, within 3 $\sigma$ of the z-resolution.}
\begin{tabular}{ |c|c| }
\hline
$|\eta|$ range & $|\Delta z|$ (cm) \\
\hline
0.0 - 0.7 & $<$ 0.4 cm \\
0.7 - 1.0 & $<$ 0.6 cm \\
1.0 - 1.2 & $<$ 0.76 cm \\
1.2 - 1.6 & $<$ 1.0 cm \\
1.6 - 2.0 & $<$ 1.7 cm \\
2.0 - 2.4 & $<$ 2.2 cm \\

\hline
\end{tabular}
\end{table}

Fake rate and the L1 Rate Reduction (min bias) plot
